% !TEX root = Document.tex
% !TeX spellcheck = fr
%% -----------------------------------
%% ---> A MODIFIER PAR L'ETUDIANT <---
%% -----------------------------------
%%
%% Commandes qui affichent le titre du document, le nom de l'auteur, etc.
\newcommand\monTitre{Méthodes d'inspection automatique d'infrastructure par robot mobile}
\newcommand\monPrenom{André Phu-Van}
\newcommand\monNom{Nguyen}
\newcommand\monDepartement{Génie Électrique}
\newcommand\maDiscipline{génie électrique}
\newcommand\monDiplome{M}        % (M)aîtrise ou (D)octorat
\newcommand\anneeDepot{2017}
\newcommand\moisDepot{décembre}
\newcommand\monSexe{M}           % "M" ou "F"
\newcommand\PageGarde{O}         % "O" ou "N"
\newcommand\AnnexesPresentes{O}  % "O" ou "N". Indique si le document comprend des annexes.
\newcommand\mesMotsClef{Liste,de,mot-clés,séparés,par,des,virgules}
%%
%%  DEFINITION DU JURY
%%
%%  Pour la définition du jury, les macros suivantes sont definies:
%%  \PresidentJury, \DirecteurRecherche, \CoDirecteurRecherche, \MembreJury, \MembreExterneJury
%%
%%  Toutes les macros prennent 4 paramètres: Sexe (M/F), Prénom, Nom, Titres
\newcommand\monJury{\PresidentJury{M}{Richard}{Gourdeau}{Ph.~D.}\\
\DirecteurRecherche{M}{Jérôme}{Le ny}{Ph.~D.}\\
\CoDirecteurRecherche{M}{David}{SAUSSIÉ}{Ph.~D.}\\
\MembreJury{M}{Liam}{Paull}{Ph.~D.}}
