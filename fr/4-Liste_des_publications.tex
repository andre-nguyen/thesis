% !TEX root = Document.tex
\chapter*{Liste des publications}

\begin{longtable}{lp{5in}}
  [Journal]     & A. Borowczyk, D.-T. Nguyen, \textbf{A. Phu-Van Nguyen}, D. Q. Nguyen, D. Saussié and J. Le Ny, \textit{Autonomous Landing of a Quadcopter on a High-Speed Ground Vehicle}. AIAA Journal of Guidance, Control and Dynamics, In Press, 2017.\\

  [Conférence]  & A. Borowczyk, D.-T. Nguyen, \textbf{A. Phu-Van Nguyen}, D. Q. Nguyen, D. Saussié and J. Le Ny, \textit{Autonomous Landing of a Multirotor Micro Air Vehicle on a High Velocity Ground Vehicle}. Proceedings of the IFAC World Congress, Toulouse, France, July 2017.\\

  [Journal]      & M. S. Ramanagopal, \textbf{A. P.-V. Nguyen} and J. Le Ny, A Motion Planning Strategy for the Active Vision-Based Mapping of Ground-Level Structures. Accepted by Transactions on Automation Science and Engineering, November 2017.
\end{longtable}

Le travail des deux premières publications à propos de l'atterrissage d'un quadricoptère sur une voiture en mouvement a été réalisé dans le cadre de la participation du \textit{Mobile Robotics and Autonomous Systems Laboratory} au DJI Challenge lors des session d'hiver et d'été 2016. N'étant pas directement liées au mémoire présent, elles sont mentionnés ici car plusieurs des méthodes de contrôle de véhicule aérien et de traitement d'images ont été apprises lors de notre participation à cette compétition.

La dernière publication intitulée \textit{A Motion Planning Strategy for the Active Vision-Based Mapping of Ground-Level Structures} avait initialement été soumise en février 2016 par M. S. Ramanagopal et J. Le Ny au journal \textit{Transactions on Automation Science and Engineering}. Lors du retour de l'évaluation par les pairs, il a été entres autres demandé d'ajouter un volet expérimental à l'article à fin de prouver la validité des algorithmes développés dans l'article. C'est à ce moment, lors de la session d'automne 2016 et au début de la session d'hiver 2017 que nous nous sommes ajouté au projet pour faire l'implémentation sur un robot physique. Au moment de l'écriture de la présente, l'article a été accepté pour publication dans une édition future du journal T-ASE.
