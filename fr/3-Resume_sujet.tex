% !TEX root = Document.tex
% !TeX spellcheck = fr
% Résumé du mémoire.
%
%   Le résumé est un bref exposé du sujet traité, des objectifs visés,
% des hypothèses émises, des méthodes expérimentales utilisées et de
% l'analyse des résultats obtenus. On y présente également les
% principales conclusions de la recherche ainsi que ses applications
% éventuelles. En général, un résumé ne dépasse pas quatre pages.
%
%   Le résumé doit donner une idée exacte du contenu du mémoire ou de la thèse. Ce ne
% peut pas être une simple énumération des parties du document, car il
% doit faire ressortir l'originalité de la recherche, son aspect
% créatif et sa contribution au développement de la technologie ou à
% l'avancement des connaissances en génie et en sciences appliquées.
% Un résumé ne doit jamais comporter de références ou de figures.
\chapter*{RÉSUMÉ}\thispagestyle{headings}
\addcontentsline{toc}{compteur}{RÉSUMÉ}

La robotique est un domaine voué à la création de machines permettant d'aider
ou de remplacer les humains lors de tâches difficiles ou dangereuses. Plus souvent
utilisés dans le secteur manufacturier ou la recherche et sauvetage, les robots
trouvent maintenant leur place dans les secteurs de l'arpentage et l'inspection
d'infrastructure civile grâce à leur mobilité et leur capacité à effectuer des
tâches répétitives. Cependant, ces robots sont souvent encore opérés à distance
et ne possèdent que de l'autonomie partielle, étant toujours incapable de prendre
des décisions eux-mêmes.

Dans ce mémoire, nous tentons de répondre à ce problème en présentant deux
méthodes permettant de guider un robot autour d'une structure pour en faire
l'inspection. Premièrement, nous attaquons le problème d'inspection au sol de
structures fermées. Fonctionnant au moyen d'un robot terrestre équipé d'une
caméra de profondeur, notre méthode permet de cartographier la surface visible
d'une structure fermée. En particulier, nous proposons un système qui cherche
à effectuer l'inspection tout en minimisant la possibilité d'erreur sur le système
de localisation du robot. Deuxièmement, nous proposons un système spécifiquement
pour l'inspection d'éoliennes. Au moyen d'un véhicule aérien
non habité et de scanneurs lasers 2D, notre méthode permet à un quadricoptère de décoller
du sol et suivre la tour pour ensuite automatiquement parcourir la surface avant
des pales d'une éolienne.

Nos systèmes sont tous les deux implémentés et démontrés dans des environnements
de simulation. L'inspection terrestre est démontrée sur un vrai robot dans un
scénario d'inspection intérieur où nous démontrons aussi comment traiter le bruit
non pris en compte dans la simulation.
Finalement, nous présentons des résultats préliminaires
sur le déploiement de l'inspection aérienne.
