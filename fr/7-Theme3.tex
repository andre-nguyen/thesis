% !TEX root = Document.tex
\Chapter{Inspection d'éoliennes par UAV}\label{sec:uav}

Dans ce chapitre nous traitons du projet de l'automatisation d'inspection d'éoliennes par un véhicule aérien autonome réalisé conjointement avec la compagnie Microdrones Canada Inc.

\section{Description du problème}

Les éoliennes doivent régulièrement être inspectées pour détecter et réparer le plus tôt possible des failles structurelles sur leurs pales qui peuvent apparaître avec le temps dû aux intempéries ou même quand elles sont frappées par la foudre. La peinture recouvrant celles-cis étant moins élastique que le composite de fibre de verre dont elles sont construites, un défaut structurel se manifeste inévitablement à la surface de la pale sous la forme de craques ou de trous. C'est pourquoi une inspection visuelle doit être régulièrement faite pour déceler et réparer toute irrégularité dans la structure avant qu'un bris catastrophique ne se produise.

En temps normal, une inspection se fait par une équipe d'une ou deux personnes devant grimper l'éolienne. D'abord, l'équipe attend que le vent fasse tourner le rotor jusqu'à ce que la pale devant être examinée pointe vers le sol. À ce moment, le frein est enclenché et l'angle d'inclinaison des pales est ajusté pour que le vent ne créé pas de couple sur le rotor. L'équipe grimpe ensuite la tour jusqu'au sommet, ce qui peut prendre une dizaine de minutes si l'ascenceur est non fontionnel, chose qui arrive fréquemment. Une fois rendu au rotor, un frein mécanique secondaire est activé pour manuellement bloquer la rotation des pales. L'équipe fait ensuite une descente en rappel le long de la pale pour faire leur inspection visuelle et tactile de la surface. Une fois l'inspection terminée le processus est recommencé pour les deux autres pales. En tout, inspecter une éolinne peut prendre entre deux et quatre heures dépendamment du nombre bris à documenter.

En plus de prendre beaucoup de temps, les inspection d'éoliennes posent un réel risque au niveau de la santé et sécurité des travailleurs nottament à cause des chutes possibles dans la tour pouvant faire plus d'une centaine de pieds de haut ou aussi par le fait de devoir naviguer des espaces confins pour se rendre au sommet de la tour \citep{Osha2017}. Ainsi, automatiser les inspections bénéfierait directement la santé des travailleurs et permetterait de sauver du temps et de l'argent.

Le but du projet est de réaliser un système capable de naviguer un quadricoptère autour des pales d'une éolienne à l'aide de capteurs peu coûteux avec un humain intervenant seulement au niveau de la prise de données. Puisque maneuvrer précisément autour des pales demanderait l'usage de capteurs dépassant le budget du projet et nécessiterait que l'UAV soit équipé de capteurs de position haute précision tel qu'un GPS différentiel, nous simplifions le problème en un problème à deux dimensions où le véhicule cheche à suivre le bord d'attaque (ou le bord de fuite) des pales. Deux approches ont été développées, l'une entièrement au moyen de scanners lasers et l'autre par caméras stereo. 

\section{Stratégie}

\subsection{Approche de loin}

\subsection{Approche en remontant la tour}


\subsection{Traitement des lasers}


\subsection{Implémentation}


\section{Simulation results}


\section{Experimental results}

\subsection{}
