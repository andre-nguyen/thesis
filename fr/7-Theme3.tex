% !TEX root = Document.tex
\Chapter{Inspection d'éoliennes par UAV}\label{sec:uav}

Dans ce chapitre nous traitons du projet de l'automatisation d'inspection d'éoliennes par un véhicule aérien autonome réalisé conjointement avec la compagnie Microdrones Canada Inc.

\section{Description du problème et méthode d'inspection}

Les éoliennes doivent régulièrement être inspectées pour détecter et réparer le plus tôt possible des failles structurelles sur leurs pales qui peuvent apparaître avec le temps dû aux intempéries ou même quand elles sont frappées par la foudre. La peinture recouvrant celles-cis étant moins élastique que le composite de fibre de verre dont elles sont construites, un défaut structurel se manifeste inévitablement à la surface de la pale sous la forme de craques ou de trous. C'est pourquoi une inspection visuelle doit être régulièrement faite pour déceler et réparer toute irrégularité dans la structure avant qu'un bris catastrophique ne se produise.

En temps normal, une inspection se fait par une équipe d'une ou deux personnes devant grimper l'éolienne. D'abord, l'équipe attend que le vent fasse tourner le rotor jusqu'à ce que la pale devant être examinée pointe vers le sol. À ce moment, le frein est enclenché et l'angle d'inclinaison des pales est ajusté pour que le vent ne créé pas de couple sur le rotor. L'équipe grimpe ensuite la tour jusqu'au sommet, ce qui peut prendre une dizaine de minutes si l'ascenceur est non fontionnel, chose qui arrive fréquemment. Une fois rendu au rotor, un frein mécanique secondaire est activé pour manuellement bloquer la rotation des pales. L'équipe fait ensuite une descente en rappel le long de la pale pour faire leur inspection visuelle et tactile de la surface. Une fois l'inspection terminée le processus est recommencé pour les deux autres pales. En tout, inspecter une éolienne peut prendre entre deux et quatre heures dépendamment du nombre bris à documenter.

En plus de prendre beaucoup de temps, les inspection d'éoliennes posent un réel risque au niveau de la santé et sécurité des travailleurs nottament à cause des chutes possibles dans la tour pouvant faire plus d'une centaine de pieds de haut ou aussi par le fait de devoir naviguer des espaces confins pour se rendre au sommet de la tour \citep{Osha2017}. Ainsi, automatiser les inspections bénéfierait directement la santé des travailleurs et permetterait de sauver du temps et de l'argent.

Certains opérateurs d'UAV offrent maintenant des services d'inspection d'éoliennes. Une caméra haute définition est montée sur un cardan stabilisateur à trois axes qui est à son tour monté sur le véhicule. Ce dernier est desfois aussi augmenté d'un télémètre permettant au pilote de savoir à quelle distance il est de la structure, chose difficile à estimer lorsque le véhicule est à 100 mètres d'altitude. Ces inspections peuvent prendre jusqu'à trois employés dont un pilote, un opérateur pour la caméra et un observateur sous l'éolienne.

Le but du projet est d'explorer les solutions possibles pour réaliser un système capable de naviguer un quadricoptère autour des pales d'une éolienne à l'aide de capteurs peu coûteux avec seulement le pilote de sécurité de présent et le moins d'information \textit{a priori} possible. Ce projet peut aussi servir de tremplin vers un projet futur dans lequel aucun humain n'interviendrait, par exemple pour une flotte d'UAV chargée d'inspecter régulièrement un parc éolien entier. Puisque maneuvrer précisément autour des pales demanderait l'usage de capteurs dépassant le budget du projet et nécessiterait que l'UAV soit équipé de capteurs de position haute précision tel qu'un GPS différentiel, nous simplifions le problème en un problème à deux dimensions où le véhicule cheche à suivre le bord d'attaque (ou le bord de fuite) des pales. Deux approches ont été développées, l'une entièrement au moyen de scanners lasers et l'autre par caméras stereo.

La mission se divise en deux phases: l'approche du rotor et le suivit de ses pales qui peuvent être gérées par une simple machine à états finis. Pour l'approche du rotor nous comparons une approche à longue distance au moyen de traitement d'images à une approche à tâtons où l'UAV remonte la tour au moyen de scanner lasers. Pour le suivit des pales, nous comparons encore une méthode par traitement d'images à une méthode de suivit à tâtons. Puisque ce projet est réalisé conjointement avec la compagnie Microdrones, le quadricoptère md4-1000 a été choisi sur lequel nous installons deux scanners lasers LeddarTech Vu8, l'un horizontal et l'autre vertical et puis une caméra stereo ZED. Il est à noter que ces capteurs ne servent qu'à la navigation de l'UAV alors que la caméra haute définition sur stabilisateur resterait le moyen par lequel prendre des photos de défauts à la surface des pales.

\begin{figure}[htp]
  \centering
  \includegraphics[width=0.5\linewidth]{images/placeholder.png}
  \caption{Systèmes de coordonnées sur le md4-1000}
  \label{fig:md41000}
\end{figure}

\begin{table}[htp]
  \centering
  \setlength{\tabcolsep}{12pt}
  \begin{tabular}[htp]{|c|l|p{10cm}|}
    \hline
    Symbole & Nom                   & Description\\\hline
    $W$     &  \textit{World}       & Suivant la convention \textit{East-North-Up} (ENU), centré au point de décollage du véhicule avec l'axe $x$ pointant vers l'est, $y$ pointant vers le nord et $z$ pointant vers le haut.\\ \hline
    $B$     &  \textit{Body}        & Encore suivant la convention ENU, l'axe $x$ pointant vers la droite, $y$ vers l'avant et $z$ vers le haut. \\ \hline
    $\mathit{BT}$  &  \textit{Body Tangent} & Repère centré sur $B$ mais compensé pour ses angles de roulis et de tangage (mais pas de lacet). En d'autres mots, le repère ${BT}$ est donc tangent à la surface de la Terre et centré sur $B$. \\ \hline
    $\mathit{C[L,R]}$ & \textit{Camera} & Suivant la convention de représentation d'images, nous avons l'axe $x$ vers la droite, $y$ vers le bas et $z$ sortant de la caméra. Les suffixes $L$ et $R$ indiquent la caméra gauche et droite respectivement. Par souci de brièveté si le repère est écrit $C$ sans préciser la caméra gauche ou droite, il indique la caméra droite.   \\ \hline
    $\mathit{L[H,V]}$ & \textit{Laser}& Les lasers aussi suivent un repère main droite avec $x$ vers l'avant, $y$ a gauche et $z$ vers le haut. Les données de distance du laser se situe tous sur son plan $XY$. Les suffixes $H$ et $V$ indique la direction du scan, horizontal et vertical, par rapport au repère $B$. \\ \hline
    $\mathit{T}$ & \textit{Turbine} & Le système de coordonnées de la turbine est centré sur le rotor avec $x$ vers la droite (vu de devant), $y$ vers l'intérieur de la nacelle et $z$ vers le haut. \\ \hline
  \end{tabular}
  \setlength{\tabcolsep}{6pt}
  \caption{Repères présents dans le système d'inspection d'éoliennes}
  \label{table:coordinate_frames}
\end{table}

\section{Approche du rotor}

\subsection{Approche de loin}

L'approche à longue distance comporte plusieurs avantages dont l'absence du besoin d'information \textit{a priori} à propos du placement des pales. Avec une vue d'ensemble de l'éolienne à inspecter, il devient possible de mesurer l'angle des pales, une information cruciale qui sera exploitée lorsque l'UAV devra choisir où se diriger lorsqu'il est près du rotor. Le deuxième avantage est qu'il est possible de mesurer l'angle de lacet relatif entre l'UAV et l'éolienne, permettant ainsi de placer le véhicule perpendiculaire à la surface des pales.

Dans cette section nous détaillons une méthode légèrement modifiée de celle de \citep{Stokkeland2015} pour détecter le centre du rotor d'une éolienne à partir d'une image. En somme les étapes restent les mêmes mais de la robustesse additionnelle est introduite au moyen d'algorithmes de regroupement, une représentation différente des pales détectées est utilisée et un filtre de Kalman non-linéaire (au lieu d'un filtre de Kalman linéaire) est proposé.

La méthode de détection du rotor se divise en 6 étapes et repose majoritairement sur des raisonnements géométriques.

%\begin{enumerate}
%  \item Prétraitement de l'image
%  \begin{enumerate}
%    \item Conversion de l'image couleur à une échelle de gris
%    \item Suppression du bruit par un filtre à convolution Gaussien
%    \item Correction du contraste par égalisation de l'histogramme
%  \end{enumerate}
%  \item Détection de contours par l'algorithme de Canny
%  \item Détection de lignes par transformée de Hough probabiliste
%  \item Trouver la tour de l'éolienne en cherchant des segments verticaux débutant au bas de l'image avec compensation de l'angle de roulis du véhicule.
%  \item Trouver les candidats de segments de pales. Ce sont les segments ayant un sommet près du plus haut point détectable de la tour.
%  \item Regroupement des segments par l'algorithme DBSCAN suivit du rejet de données aberrantes par une procédure de vote.
%  \item Moyenner les segments de pales et trouver le rotor par l'intersection du prolongement des pales.
%\end{enumerate}

\subsubsection{Étape 1: Prétraitement de l'image} Avant de commencer la détection, quelques corrections à l'image doivent être apportées. Puisque la procédure ne requiert pas de données de couleur, une conversion RGB à échelle de gris est effectuée au moyen de la transformée

\begin{align}
 Y \leftarrow 0.299 R + 0.587 G + 0.114 B
 \label{eq:rgb2gray}
\end{align}

où $Y$ est le niveau de gris du pixel entre $0$ et $255$ ce qui donne comme résultat l'image de la Figure \ref{fig:detection_pretraitement} (B). Pour augmenter le contraste et corriger les erreures d'exposition de la caméra, une égalisation d'histogramme est appliquée. Pour l'image originale $I$ et l'image égalisée $H$ que l'on peut voir dans la Figure \ref{fig:detection_pretraitement} (C), nous avons

\begin{align}
  p_n &= \frac{\text{\# de pixels d'intensité } n }{\text{\# de pixels total}}, \ \ n = 0, \ldots, 255 \\
  H_{i,j} &= \text{floor}(255 \sum_{n=0}^{I_{i,j}} p_n)
  \label{eq:egalisation_histogramme}
\end{align}

Où floor est l'opérateur arondissant à la baisse à l'entier le plus proche. Pour éliminer la présence de bruit et atténuer les textures à haute fréquence pouvant donner lieu à une sur détection de lignes lors de la transformée de Hough, nous appliquons un filtre moyenneur 3x3 par convolution. Pour l'image floutée $F$ nous avons donc

\begin{align}
  F = \Bigg(\frac{1}{9}
    \begin{bmatrix}
      1 & 1 & 1\\
      1 & 1 & 1\\
      1 & 1 & 1
    \end{bmatrix}
  \Bigg) * H
  \label{eq:boxfilter}
\end{align}

Pour les convolutions sur les bords de l'image nous répétons la valeur du pixel du bord pour les données manquantes dans l'opération de convolution.

\begin{figure}[htp]
\centering
\begin{minipage}{0.4\textwidth}
  \centering
  \includegraphics[width=\linewidth]{images/placeholder.png}
  (A)
\end{minipage}
\begin{minipage}{0.4\textwidth}
  \centering
  \includegraphics[width=\linewidth]{images/placeholder.png}
  (B)
\end{minipage}
\begin{minipage}{0.4\textwidth}
  \centering
  \includegraphics[width=\linewidth]{images/placeholder.png}
  (C)
\end{minipage}
\begin{minipage}{0.4\textwidth}
  \centering
  \includegraphics[width=\linewidth]{images/placeholder.png}
  (D)
\end{minipage}
\caption{(A) Image originale (B) Image en tons de gris (C) Image avec égalisation d'histogramme (D) Image floutée par un filtre moyenneur}
\label{fig:detection_pretraitement}
\end{figure}

\subsubsection{Étape 2: Détection de contours} La détection de contours se fait au moyen de l'algorithme de \citep{Canny1986} qui à son tour se fait en une série de 4 étapes:

\begin{enumerate}
  \item Un filtre Gaussien par convolution est appliqué pour atténuer le bruit. Pour générer un masque Gaussien $H$ en deux dimensions il suffit de suivre l'équation:
  \begin{align}
    H_{ij}= \frac{1}{2\pi\sigma^2}\exp \left(-\frac{(i-(k+1))^2+(j-(k+1))^2}{2\sigma^2} \right) ; 1 \leq i, j \leq (2k + 1)
  \end{align}
  où $k$ est la taille du masque, $\sigma$ est l'écart type de la distribution et $i$ et $j$ sont les indexes de rangée et de colonne respectivement. Pour $k = 5$ et $\sigma = 1.4$ nous avons
  \begin{align}
    I_{\text{floue}} = \frac{1}{159} \begin{bmatrix}
      2 & 4 & 5 & 4 & 2\\
      4 & 9 & 12 & 9 & 4\\
      5 & 12 & 15 & 12 & 5\\
      4 & 9 & 12 & 9 & 4\\
      2 & 4 & 5 & 4 & 2
    \end{bmatrix} * I_{\text{originale}}
  \end{align}
  \item Pour chaque pixel le gradient $G$ et l'angle du gradient $\Theta$ est calculé à partir des quantitées $G_x$ le gradient horizontal et $G_y$ le gradient vertical suivant les équations:
  \begin{align}
    G_x =  \begin{bmatrix} -1 & 0 & 1 \\
    -2 & 0 & 2 \\
    -1 & 0 & 1
  \end{bmatrix} * I_{\text{floue}}
  \end{align}
  \begin{align}
    G_y = \begin{bmatrix}
    -1 & -2 & -1\\
    0 & 0 & 0 \\
    1 & 2 & 1
  \end{bmatrix} * I_{\text{floue}}
  \end{align}
  \begin{align}
    G = \sqrt{G_x^2 + G_y^2}
  \end{align}
  \begin{align}
    \Theta = \atantwo(G_y^2, G_x^2)
  \end{align}
  \item Une suppression des non-maxima est appliquée. Pour un pixel de l'image gradient $G_{ij}$ nous comparons la valeurs aux deux pixels adjacents dans les directions $\pm\Theta$, si la valeur de $G_{ij}$ est plus grande que les deux autres, il est gardé, sinon il est mit à $0$.
  \item Un double seuil, l'un haut $T_h$ et l'autre bas $T_b$, est appliqué par hystérésis pour atténuer les pixels de contours trop faibles. Un pixel $G_{ij}$ plus fort que $T_h$ est marqué comme un pixel au contour fort. Les pixels plus faibles que $T_b$ sont supprimés (mise à $0$). Les pixels entre les deux sont gardés seulement s'ils sont adjacents à un pixel fort. Canny suggère que le ratio $T_b$ à $T_h$ soit entre $1:2$ et $1:3$.
\end{enumerate}

Le résultat de la procédure de Canny est visible dans la Figure \ref{fig:canny}.

\begin{figure}[htp]
  \centering
  \includegraphics[width=0.5\linewidth]{images/placeholder.png}
  \caption{Résultat de l'application du filtre de Canny avec $T_b = 100$ et $T_h = 300$}
  \label{fig:canny}
\end{figure}

Outre la méthode de Canny, il existe des méthodes modernes plus performantes que celle de Canny, par exemple \citep{Xie2015} proposent un réseau de neuronnes entièrement convolutionel capable de détecter beaucoup mieux les contours, et ce, sans avoir à manuellement ajuster des paramètres.

\incomplete  Expliquer que ça marchait trop bien?

\subsubsection{Étape 3: Détection de lignes}

Une fois les contours trouvés, nous pouvons les utiliser pour détecter des segments de droite dans l'image. Dans le cas où des parties du véhicule sont visibles dans l'image, il suffit d'appliquer un masque avant de débuter cette étape. Pour trouver les segments, nous appliquons la transformée de Hough probabiliste pour détecter des segments de droites dans l'image de contours.

\subsubsection{Étape 4: Recherche de la tour}
\label{subsubsec:approach4}

Stokkeland explique que la partie la plus facile à détecter est la tour puisqu'elle est toujours verticale et répond fortement à la détection de lignes. Pour la trouver, il suffit de chercher une ligne verticale (compensée pour l'angle de roulis de l'UAV) dont l'un des sommets repose dans le base de l'image. Puisqu'il arrive parfois que la détection de ligne défait la tour en plusieurs segments, une fois le segment initial choisit on recherche aussi des prolongements de ce segment à une certaine distance du plus haut sommet.

\subsubsection{Étape 5: Recherche des pales}

À la fin de l'étape 4 nous avons un ensemble de segments appartenant à la tour. À partir du plus haut point de ces segments nous recherchons tout autre segment ayant un sommet dans un certain rayon du point. Ces segments sont considérés en tant que candidats à être des segments d'une pale.

Nous divergeons de la méthode proposée par Stokkeland où il tente de regrouper les segments selon l'angle par rapport à la tour; Au travers d'une procédure vorace simple où un nouveau groupe est créé lorsqu'un segment est à un angle au dessus d'un certain seuil du groupe courant. Nous notons que ceci fonctionne acceptablement sur le jeu de données à Stokkeland où le ciel est clair et son éolienne est sur une colline près de la mer, mettant donc l'horizon bien en dessous du rotor. Lors de nos essais tant en simulation que sur le terrain, nous avons noté que l'horizon était très proche de rotor donnant lieu à une grande présence de segments près du rotor faussant ainsi les résultats de la recherche de pales. C'est pourquoi nous implémentons l'étape 5 avec un algorithme de groupage formel qui inclut la résistance au bruit de mesure nommé \textit{Density-based spatial clustering of applications with noise} (DBSCAN) proposé par \citep{Ester1996}.

La particularité de DBSCAN par rapport à d'autres algorithmes de partionnage tel que $k$-means est que nous n'avons pas à fournir le nombre $k$ de partitions recherchées. À la place, nous fournissons que deux paramètres $\epsilon$ (eps) la distance d'un point à son entourage et \matht{MinPts} le nombre minimum de points pour qu'une partition soit formée. Pour chaque segment, nous projettons leur angle par rapport à la tour sur un cercle unitaire. Ceci permet d'exécuter DBSCAN dans un espace euclidien pour regrouper les segments adjacents.

Une fois les groupes formés, Stokkeland utilise une stratégie de vote pour éliminer les fausses détections. La moyenne des angles est calculée et une procédure de vote débute s'il y a plus de 2 groupes. Nous choisissons 2 au lieu de 3 tel que proposé par Stokkeland pour prendre en compte le cas où 1 pale est vis-à-vis la tour. Sachant que les éoliennes ont toujours 3 pales espacées de $120$ degrés, chaque groupe calcule sa distance angulaire par rapport aux autres et vote pour les groups n'étant pas à $120$ degrés de eux-mêmes. Dans le cas où DBSCAN ressort que 2 groupes, la direction de la troisième pale est approximée à 120 degrés des deux autres pales.

\subsubsection{Étape 6: Recherche du rotor}

L'étape 5 nous a donc fournis 3 groupes de segments représentant chacuns les pales de l'éolienne. Le centre do rotor peut donc être calculé en moyennant chaque groupe puis en prenant la moyenne de l'intersection des prolongements de chaque segment.

Une fois le point dans l'image trouvé, il suffit d'appliquer le modèle de caméra sténopé et sa matrice de paramètre intrinsèques $K$ pour obtenir un vecteur unitaire $u_C$ dans le repère caméra pointant vers le rotor. Puisque la distance entre l'UAV et le rotor est beaucoup plus grande que la distance entre la caméra et le centre du véhicule nous pouvons la négliger. Ce qui implique que nous pouvons appliquer quelques rotations pour remettre ce vecteur unitaire dans le système d'axes $W$ (centré sur $B$) à partir de $C$ pour obtenir $u_W$.

\color{red}
Mettre les équations de rotation et expliquer ce que stokkeland a fait
\color{black}

Stokkeland met ensuite

\color{red}
Text à propos du filtre EKF à la place du KF pour track le centre du hub
\color{black}

\subsection{Approche en remontant la tour}

Dans cette méthode le véhicule est placé à proximité de la tour de tel sorte que le scanner laser puisse capter la forme de la tour et de la discontinuité introduite à l'endroit où la tour rencontre le rotor. Cette méthode est plus simple et plus fiable que l'approche visuelle mais elle repose sur deux suppositions:
\begin{enumerate}
  \item Le véhicule est bien placé par l'opérateur devant l'éolienne de tel sorte les capteurs seront perpendiculaires à la surface des pales.
  \item Aucune pale n'est placée ni proche, ni devant la tour.
\end{enumerate}

Le principe opérationel est simple et illustré dans la Figure \ref{fig:approach}, le scanner laser horizontal est utilisé pour centrer l'UAV devant la tour alors que le scanner vertical est utilisé pour détecter le rotor. Pour ce faire il faut tout d'abord projetter le vecteur de distances à un nuage de points 3D au moyen de l'équation \ref{eq:scan2cloud} où $r$ est une distance et $\theta$ est l'angle correspondant à celle-ci.

\begin{align}
  p_{L} = \begin{bmatrix}x \\ y \\ z\end{bmatrix} = \begin{bmatrix}r\cos \theta \\ r \sin \theta \\ 0 \end{bmatrix}
    \label{eq:scan2cloud}
\end{align}

\begin{figure}[htp]
  \centering
  \includegraphics[width=0.5\linewidth]{images/placeholder.png}
  \caption{Principe d'opération de l'approche du rotor par laser.}
  \label{fig:approach}
\end{figure}

L'équation \ref{eq:scan2cloud} nous donne des points dans le repère du laser correspondant, puisque les matrices de transformées homogènes $T_{LH}^B$, $T_{LV}^B$ ainsi que la rotation $R_B^{BT}$ sont connues, il suffit de transformer chaque $p_L$ au repère ${BT}$ suivant l'équation \ref{eq:pointcloud_bodytangent}.

\begin{align}
  p_{BT} = \begin{bmatrix}
    R_B^{BT} & 0 \\
    0 & 1
  \end{bmatrix} T_{L}^B p_L
  \label{eq:pointcloud_bodytangent}
\end{align}

\color{red}
test
\color{black}

\section{Suivit des pales}

\subsection{Suivit par nuage de points}

\subsection{Suivit par scanner laser 2D}

\subsection{Suivit visuel pur}

\section{Implémentation}

\subsection{Résultats en simulation}


\subsection{Résultats de tests sur le terrain}


\subsection{}
