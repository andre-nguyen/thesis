% !TEX root = Document.tex
%   Dans l'introduction, on présente le problème étudié et les buts
% poursuivis. L'introduction permet de faire connaître le cadre de la
% recherche et d'en préciser le domaine d'application. Elle fournit
% les précisions nécessaires en ce qui concerne le contexte de
% réalisation de la recherche, l'approche envisagée, l'évolution de
% la réalisation. En fait, l'introduction présente au lecteur ce
% qu'il doit savoir pour comprendre la recherche et en connaître la
% portée.
\Chapter{INTRODUCTION}\label{sec:Introduction}  % 10-12 lignes pour introduire le sujet.
Le domaine de la robotique s'applique à ce qui est communément appelé les trois \emph{D} de la robotique \emph{Dull}, \emph{Dirty} et \emph{Dangerous}, c'est-à-dire les tâches ennuyantes, sales et dangereuses. L'inspection d'infrastructures civiles rentre dans la première et la dernière de celles-ci. Ennuyante car elle implique la cueillette répétitive de données et dangereuse car elle implique parfois l'obligation de grimper dans des structures hors de portée ou difficiles à atteindre.

En général, le but de ces inspections est multiple: outre la recherche de défauts dans l'infrastructure, elles peuvent aussi servir à suivre de près l'évolution d'un chantier de construction, faire de l'arpentage ou planifier de futurs projets de développement. L'intérêt de faire exécuter l'inspection par un robot autonome est d'une part d'accélérer la collecte de données par un robot plus mobile qu'un humain, et d'autre part de réduire les coûts liés à la collecte d'informations.

%\section{Problématique considérée}

%Dans ce projet nous distinguons deux types d'inspections: les inspections de structures au sol effectuées par un robot terrestre non-habité (UGV) et les inspections en en hauteur effectuées par un robot aérien non-habité (UAV). Les deux cas comportent des similarités notamment par leurs besoin de systèmes de positionnement et de systèmes de planification de trajectoire.

\section{Objectif de recherche}

L'objectif principal de la recherche est de développer des méthodes d'inspection d'infrastructures adaptées aux robots terrestres non-habités (UGV) ainsi qu'aux robots aériens non-habités (UAV). Alors que la navigation du robot se fait entièrement de façon autonome, il est attendu qu'un opérateur intervienne seulement lors de la phase d'analyse des données et dans le cas d'un UAV qu'il intervienne dans la prise de photos.

\subsection{Inspections au sol}
Dans un premier lieu, nous nous attardons au problème de la création de cartes 3D précises de structures au sol. Ces cartes peuvent être utilisées pour une variété d'applications telles que la réalité virtuelle \citep{googlevr2017}, la navivation de véhicules autonomes \citep{deepmap2017} et la surveillance du progrès d'un chantier de construction \citep{Omar2018}. Cette dernière application, étant bien établie en industrie, se fait habituellement par l'une des trois méthodes suivantes: la photogrammétrie ou vidéogrammétrie, les scanner lasers et les images de profondeur. Chacune d'entres elles demande qu'un opérateur humain parcourt le chantier en cherchant les meilleurs points de vues avec les capteurs appropriés en main pour la collecte de données. En particulier, dû à la relative courte portée de capteurs d'images de profondeur, elles demandent aux opérateurs de parcourir une plus grande distance et plus de prises de vue.

C'est dans ce contexte que nous proposons un scénario où un UGV fait l'inspection de la partie visible d'une structure au sol, au moyen d'un capteur de profondeur, sans aucune information \textit{a priori}. Pour commencer, nous nous basons sur des algorithmes de navigation et de localisation existants tels que les champs de potentiel, le Simultaneous Localization and Mapping (SLAM) visuel et la fusion d'odométrie de roues avec une centrale inertielle (IMU) pour effectuer le contrôle du véhicule. L'objectif est d'avoir une méthode de génération de trajectoires en temps réel qui, d'une part facilite la précision du système SLAM et d'autre part garantir la couverture entière de la surface de l'édifice.

\subsection{Inspections aériennes}
Dans un second lieu, nous considérons le cas où un UAV fait l'inspection de la surface d'une éolienne. Cette partie du projet, réalisée conjointement avec la compagnie Microdrones Canada, s'inscrit dans le cadre d'un plus large projet de développement qui vise à développer une solution complète d'inspection d'éoliennes par UAV autonome permettant de réduire le temps nécessaire à une inspection. Tout d'abord, une inspection sert à repérer des défauts dans la structure d'une éolienne. Laissés seuls, ces défauts peuvent se traduirent en perte d'efficacité et dans le pire cas, en un bris fatal.

En temps normal, une inspection implique qu'un employé fasse tourner les pales de l'éolienne pour en positionner une vers le bas. L'employé grimpe en haut de la tour pour ensuite faire une descente en rappel le long de la pale pour inspecter visuellement et tactilement la surface de la pale. Ce processus devant être répété pour chaque pale, il peut prendre entre 2 et 4 heures dépendamment de la quantité de défauts à documenter. Plus récemment, nous notons un immense essor dans le nombre de compagnies offrant des services d'inspection par UAV qui ne requièrent pas l'ascention de la tour. Étant plus rapide, plus sécuritaire et moins dispendieuse, cette méthode est rapidement adoptée par l'industrie de l'énergie. Un rapport publié en 2015 par la firme d'études de marché Navigant Research prédit que le revenu global lié aux ventes d'UAV et de services d'inspection atteindra près de 6 milliards USD en 2024 \citep{navigant2015}.

Le problème de ces systèmes est qu'ils demandent habituellement une équipe d'au moins deux personnes dont un pilote habile et un opérateur de caméra pour fonctionner. Le but général du projet est de rendre l'utilisation de l'UAV plus accessible afin de retirer le besoin d'un pilote et réduire la taille de l'équipe à une seule personne. Il faut noter que le but n'est pas encore de remplacer complètement une inspection manuelle par un humain. En effet, puisque la peinture reposant sur la surface d'une pale est moins flexible que le composite de la structure qui se retrouve en-dessous, un défaut dans la structure implique une fissure dans la peinture. L'inverse n'étant pas nécessairement vrai, une inspection visuelle par UAV devra parfois demander un deuxième passage de près par un humain pour confirmer l'existence du défaut.

En somme, le but final du projet est de produire un système autonome plus facile d'utilisation pour l'inspection d'éoliennes et pouvant être déployé par un seul opérateur.

\section{Plan du mémoire}

Avant tout, nous commençons par une revue de littérature dans le Chapitre \ref{sec:RevLitt} pour passer en revue quelques concepts importants se rapportant à la vision par ordinateur et la navigation pour robots autonomes. Par la suite, nous présentons l'essentiel de notre travail en trois volets. Dans le Chapitre \ref{sec:capteurs} nous faisons l'étude des capteurs existant sur le marché avec une comparaison des avantages et des désavantages de ceux-ci par rapport à l'inspection d'infrastructure. Au Chapitre \ref{sec:ugv}, nous présentons notre travail réalisé dans le cadre de nos recherches sur l'inspection d'infrastructures par UGV qui s'est culminé en un article soumis en mars 2016 au journal \textit{Transactions on Automation Science and Engineering}. En dernier lieu, au Chapitre \ref{sec:uav}, nous présentons notre travail réalisé lors du projet Mitacs Accelerate IT08812 en collaboration avec la compagnie Microdrones à propos de l'inspection d'éoliennes par drone autonome.


%Débutant par les bases d'un véhicule autonome, nous survolons les méthodes d'estimation d'état tant au niveau de l'estimation de l'attitude de notre véhicule qu'au niveau de la position de notre véhicule par rapport à un référentiel quelconque près de la structure à inspecter. Ensuite nous examinons le travail existant relatif à la planification de trajectoires et de couverture de surface. Les ouvrages examinés nous permettent de mieux situer la contribution exacte du volet UGV de notre projet, dans le corps de travail existant. Puisqu'il doit y avoir un moyen pour nos véhicules d'exécuter les dites trajectoires, nous survolons rapidement certaines méthodes de navigation pour le suivi de trajectoires et le contrôle des véhicules. Finalement, nous terminons la revue de littérature par une vue d'ensemble sur les méthodes de cartographie pour la reconstruction 3D.

% Sachant la géométrie approximative de la structure nous pouvons en profiter pour développer un système se basant sur une simple machine à états finis. De plus, les requis du projet étants relativement ouverts, nous examinons aussi les types de charge utile permettant l'accomplissement de l'inspection pour expliquer notre choix final d'une combinaison de télémètres lasers et d'une caméra RGB. Le fonctionnement du système est prouvé dans une variété d'environnements de simulation réalistes et certaines fonctions sont démontrés hors ligne sur des ensembles de données prises lors d'un vol manuel.
