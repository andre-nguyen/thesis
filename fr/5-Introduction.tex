% !TEX root = Document.tex
%   Dans l'introduction, on présente le problème étudié et les buts
% poursuivis. L'introduction permet de faire connaître le cadre de la
% recherche et d'en préciser le domaine d'application. Elle fournit
% les précisions nécessaires en ce qui concerne le contexte de
% réalisation de la recherche, l'approche envisagée, l'évolution de
% la réalisation. En fait, l'introduction présente au lecteur ce
% qu'il doit savoir pour comprendre la recherche et en connaître la
% portée.
\Chapter{INTRODUCTION}\label{sec:Introduction}  % 10-12 lignes pour introduire le sujet.
Le domaine de la robotique s'applique à ce qui est communément appelé les trois \emph{D} de la robotique \emph{Dull}, \emph{Dirty} et \emph{Dangerous}, c'est-à-dire les tâches ennuyantes, sales et dangereuses. L'inspection d'infrastructure civile rentre dans la première et la dernières de celles-cis. Ennuyantes car elle implique la prise répétitive de photos et dangereuse car elle implique parfois l'obligation de grimper dans des structures hors de portée our difficiles à atteindre.

En général, le but de ses inspections est multiple, outre la recherche de défauts dans l'infrastructure, elles peuvent aussi servir à suvire de près l'évolution d'un chantier de construction, faire de l'arpentage et planifier de futurs projets de développement. L'intérêt de faire exécuter l'inspection par un robot autonome est d'une part d'accélérer la collecte de données par un robot plus mobile qu'un humain et d'autre part de réduire les coûts liées à la collecte d'information.

Dans ce projet nous distinguons deux types d'inspections: les inspections de structures au sol effectuées par un robot terrestre non-habité (UGV) et les inspections en en hauteur effectuées par un un robot aérien non-habité (UAV). Les deux cas comportent des similarités notamment par leurs besoin de systèmes de positionnement et de systèmes de planification de trajectoire.

\section{Survol des méthodes d'inspection autonomes}

Avant de choisir une méthode d'inspection, il faut tout d'abord définir la métrique par laquelle nous évaluons 

\section{Problématique considérée}

\subsection{Objectif de recherche}

\section{Méthodes de positionnement}

Nous faisons usage de trois types de systèmes de positionnement soit:
\begin{enumerate}
  \item Les systèmes relatifs tel que l'odométrie provenant des roues d'un robot.
  \item Les systèmes absolus tel que GPS ou GLONASS reposant sur les données de satelittles.
  \item Les systèmes relatifs à une structure tel que le \textit{Simultaneous Localization And Mapping} (SLAM).
\end{enumerate}

De base, tout système de positionnement modernes sont développés autour d'une centrale inertielle, un module contenant un accéléromètre pour mesurer l'accélération du véhicule, un gyroscope pour mesurer la vitesse angulaire et souvent aussi d'un magnétomètre pour mesurer le champ magnétique de la terre. Ces capteurs nous permettent au minimum d'estimer l'attitude de notre véhicule dans le monde

\section{Méthodes de navigation}
