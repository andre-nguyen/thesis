% !TEX root = Document.tex
%!TeX spellcheck = fr
% Remerciements
%
%   Grâce aux remerciements, l'auteur attire l'attention du lecteur
% sur l'aide que certaines personnes lui ont apportée, sur leurs
% conseils ou sur toute autre forme de contribution lors de la
% réalisation de son mémoire. Le cas échéant, c'est dans cette section
% que le candidat doit témoigner sa reconnaissance à son directeur de
% recherche, aux organismes dispensateurs de subventions ou aux
% entreprises qui lui ont accordé des bourses ou des fonds de
% recherche.
\chapter*{REMERCIEMENTS}\thispagestyle{headings}
\addcontentsline{toc}{compteur}{REMERCIEMENTS}
%
%Je remercie mon chien Dopey pour le support émotionel à travers tout ce périple. Pour avoir toujours cru en moi et m'avoir supporté en temps de besoin.

Je tiens d'abord à remercier mes amis et collègues de laboratoire Alexandre Borowczyk, Dang Quang Nguyen, Duc Tien Nguyen et Gabriel Guilbert pour les fructueuses discussions théoriques et surtout pour avoir participé avec moi au DJI Challenge 2016. Ce fut la compétition d'ingénierie la plus difficile à laquelle j'ai participé durant mon cheminement académique et ce fut un honneur de l'avoir fait avec vous.

Je remercie aussi mes collègues Olivier Gougeon, Justin Cano, Jérémie Pilon, Catherine Massé et Juliette Tibayrenc pour les bons moments passés dans le Laboratoire de Robotique Mobile et des Systèmes Autonomes.

\newcommand{\RNum}[1]{\uppercase\expandafter{\romannumeral #1\relax}}

Je remercie Laurence Lebel, Alexandre Marceau-Gozsy, Simon Dufour, Mélanie Harvey, Simon Bourgault-Côté, Yoann Arpin, Marie Tardif-Drolet, Gabriel Brassard, Khaldoun El-Hajj, Marc Castanet, Alessandro Scola, Hubert Courteau-Godmaire, Félix Amyot et toutes les autres personnes que j'ai pu côtoyer lors de la construction de la voiture solaire Esteban \RNum{6}. Les deux années que j'ai passé dans l'équipe avec vous étaient essentielles à ma formation d'ingénieur et m'ont amenées à étendre mes connaissances bien au-delà des sujets de l'informatique.

Je remercie Antoine Mignon, Marc-André Ruel, David Thibodeau, Laurier Loiselle, Antonio Sanniravong, Constant Rietsch, Jean-Aleandre Barszcz, Quentin Gili, Alexandre St-Onge, David Binet et toutes les autres personnes qui on aidé de près ou de loin à la formation et l'opération de l'équipe du multirotor autonome Élikos. Je suis fier d'avoir bâti avec vous un projet d'envergure dont la pérenité est maintenant assurée par une fortre tradition d'excellence. Mes deux ans en tant que directeur du projet (et nos deux premières place à l'\emph{International Aerial Robotics Competition}) ont été un facteur décisif dans ma poursuite du sujet de la robotique aérienne aux cycles supérieurs.

%Je remercie aussi Maude Carrier, le temps que l'on a passé ensemble dans le cours de Traitement des Signaux et d'Images fût bien amusant.

Finalement j'aimerais remercier mes directeurs de recherche Jérôme Le Ny et David Saussié pour avoir accepté de superviser mes travaux de maîtrise et pour m'avoir aidé sans relâche à travers tout mon parcours de maîtrise.
