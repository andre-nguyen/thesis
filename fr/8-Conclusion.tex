% !TEX root = Document.tex
%!TeX spellcheck = fr
\Chapter{CONCLUSION}\label{sec:Conclusion}

Dans ce mémoire, nous avons présenté deux méthodes, pour l'inspection de diverses infrastructures civiles, l'une par UGV et l'autre par UAV. À travers ceci, nous étudions aussi l'utilisation d'une variété de capteurs permettant la navigation d'un robot dans un environnement inconnu. Dans ce chapitre, nous faisons un survol des méthodes proposées et nous proposons des pistes pour la poursuite du travail présenté.

%%
%%  SYNTHESE DES TRAVAUX
%%
\section{Synthèse des travaux}

Le système d'inspection par UGV permet de cartographier la surface visible d'une structure tout en minimisant l'erreur de localisation sur le module de SLAM utilisé. L'UGV perçoit son environnement au moyen d'un bras articulé sur lequel nous installons une caméra de profondeur. La carte construite par la caméra permet ensuite d'appliquer la méthode des champs de potentiel pour déplacer le robot à l'endroit désiré. Notre stratégie consiste en une exploration de périmètre permettant d'effectuer une fermeture de boucle le plus tôt possible. Une fois la boucle fermée et la contrainte sur le graphe de poses imposé, l'exploration des cavités débute pour compléter le modèle en construction. Les cavités sont détectées par une analyse des voxels d'une OctoMap pour décerner les frontières entre l'espace connu et l'espace inconnu. Nous prouvons la validité de la méthode au travers de multiples expériences tant en simulation qu'en expérimentation réelle. Le travail a eu pour point culminant la publication d'un article dans le journal \emph{IEEE Transactions on Automation Science and Engineering} \citep{Ramanagopal2017}.

Nous proposons aussi un système d'inspection par UAV permettant de survoler la surface des pales d'une éolienne à l'arrêt. Au moyen de deux lidars 2D, une méthode de contrôle PID est appliquée pour centrer le véhicule sur la partie de l'éolienne sous inspection. Une simple machine à états dirige l'UAV pour décoller du sol et inspecter une grande partie de la surface avant de la structure. La méthode est validée à travers de multiples simulations dans l'environnement Gazebo. Cependant, plus de travail serait requis pour terminer l'implémentation sur un véhicule réel. Nous étudions aussi l'utilisation de vision par ordinateur pour la détection et l'approche de l'éolienne. Les résultats expérimentaux indiquent que la poursuite de cette méthode demanderait l'exploration d'algorithmes entièrement différents.


%%
%%  LIMITATIONS
%%
\section{Limitations de la solution proposée et améliorations futures}\label{sec:Limitations}

Dans les sections \ref{subsec:ugv_noise_sim} et \ref{sec:ugv_conclusion}, nous avons fait un survol des cas d'échec possible de la méthode d'inspection par UGV. En particulier, notre algorithme est sensible aux irrégularités présentes à la surface de la structure, telles que des fenêtres ou des trous dans les murs. En présence de cette sorte de défaut, le caméra de profondeur peut capter une partie de l'intérieur de la structure, ce qui fausse le calcul de la prochaine pose objectif lors du suivi de mur. Bien qu'un seuillage sur le nuage de points est possible pour mitiger ce problème, il peut arriver qu'une partie des points qui nous intéressent soient perdus dans le processus. Pour mitiger le problème il faudrait développer un moyen intelligent de segmenter le nuage de points capté pour n'inclure que la surface de la structure. Ce problème sera particulièrement important à résoudre pour déployer notre robot dans des missions d'inspection de chantiers où les surfaces de la structure seront incomplètes. 

Dans la section \ref{uav:results_conclusion_future_work}, nous avons discuté des limites du système d'inspection par UAV. Le projet a été réalisé avec une simplification de la tâche où, seul le bord d'attaque des pales de l'éolienne devait être inspectées. Un système complet devrait être capable d'inspecter toutes les facettes d'une pale incluant les deux côtés, le bord d'attaque et le bord de fuite. De plus, la conception du véhicule multi-rotor devrait être revue pour éliminer le moteur dans le champ de vision de la caméra et pour permettre à la caméra de pointer vers le haut afin d'inspecter le dessous des pales. 

Une axe de recherche intéressante pour la poursuite de ce travail serait de résoudre le problème de l'approche de loin sachant par exemple seulement les coordonnées GPS de l'éolienne. Ceci serait la prochaine étape avant d'attaquer le développement d'une flotte d'UAV assurant l'entretien d'un parc éolien entier en effectuant des inspections en parallèles. Une autre piste de recherche pour la continuation de ce projet serait l'inspection d'éoliennes en mouvement.

%%
%%  AMELIORATIONS FUTURES
%%
%\section{Améliorations futures}
%Texte.
