% !TEX root = Document.tex
\Chapter{REVUE DE LITTÉRATURE}\label{sec:RevLitt}

Bien que les deux types d'inspections traités dans cet ouvrage sont exécutées sur des véhicules entièrements différents, les concepts utilisées lors de ces opérations se recoupent énormément.

\section{Méthodes de positionnement}

Nous faisons usage de trois types de systèmes de positionnement soit:
\begin{enumerate}
  \item Les systèmes \textit{dead reckoning} fonctionnant par l'intégration d'une mesure à travers le temps tel que l'odométrie provenant des roues d'un robot.
  \item Les systèmes absolus tel que GPS ou GLONASS reposant sur les données de satelittles.
  \item Les systèmes relatifs à une structure tel que le \textit{Simultaneous Localization And Mapping} (SLAM).
\end{enumerate}

De base, tout système de positionnement modernes sont développés autour d'une centrale inertielle, un module contenant un accéléromètre pour mesurer l'accélération du véhicule, un gyroscope pour mesurer la vitesse angulaire et souvent aussi d'un magnétomètre pour mesurer le champ magnétique de la terre. Ces capteurs nous permettent au minimum d'estimer l'attitude de notre véhicule dans le monde

\section{Génération de trajectoires et couverture de surfaces}

En général, le but des générateurs de trajectoire est d'optimiser une certaine métrique permettant de décider à quel endroit il faut placer le véhicule et son capteur pour faire l'inspection de la structure.

Tout d'abord, considérant le cas d'une mission sans information \textit{a priori}, il s'agit d'une mission d'exploration où la génération de trajectoire doit se faire itérativement en temps réel pour envoyer le robot aux confins de l'espace connu. On tente en fait de répondre à la question:

\begin{quote}
  Given what you know about the world, where should you move to gain as much new information as possible? (Sachant ce que nous savons à propos du monde, où devrions nous aller pour gagner le plus d'information possible?) \citep{Yamauchi1997}.
\end{quote}

Pour ce faire Yamauchi introduit le concept de \textit{frontier-based exploration} qui cherche à guider un robot vers la frontière entre l'espace connu et ouvert et l'espace inconnu. Suivant cette idée, plusieurs chercheurs proposent des fonctions de coût à minimiser permettant de choisir à quel endroit de la frontière explorer. Wirth et Pellenz proposent de subdiviser une carte 2D en celulles pour lesquelles une fonction de coût est calculée en prenant en compte non seulement la plus courte distance par rapport au robot mais aussi la distance par rapport à l'obstacle le plus proche. Chaque nouvelle position objectif minimise ainsi la distance pour s'y rendre mais aussi le risque encouru par le robot \citep{Wirth2007}. Bircher et al. proposent une méthode d'exploration 3D 



À l'opposée, une trajectoire globalement optimale peut être générée au préalable si une carte est déjà disponible. Pour une mission 



\section{Méthodes de navigation}

\section{Reconstruction 3D}
