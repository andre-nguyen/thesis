% !TEX root = Document.tex
\Chapter{REVUE DE LITTÉRATURE}\label{sec:RevLitt}
Texte.

\section{Méthodes de positionnement}

Nous faisons usage de trois types de systèmes de positionnement soit:
\begin{enumerate}
  \item Les systèmes \textit{dead reckoning} fonctionnant par l'intégration d'une mesure à travers le temps tel que l'odométrie provenant des roues d'un robot.
  \item Les systèmes absolus tel que GPS ou GLONASS reposant sur les données de satelittles.
  \item Les systèmes relatifs à une structure tel que le \textit{Simultaneous Localization And Mapping} (SLAM).
\end{enumerate}

De base, tout système de positionnement modernes sont développés autour d'une centrale inertielle, un module contenant un accéléromètre pour mesurer l'accélération du véhicule, un gyroscope pour mesurer la vitesse angulaire et souvent aussi d'un magnétomètre pour mesurer le champ magnétique de la terre. Ces capteurs nous permettent au minimum d'estimer l'attitude de notre véhicule dans le monde

\section{Génération de trajectoires et couverture de surfaces}

\section{Méthodes de navigation}

\section{Reconstruction 3D}
