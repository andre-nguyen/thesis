% !TEX root = Document.tex
\Chapter{Inspection d'infrastructure par UGV}
\label{sec:ugv}

Dans cette section nous présentons notre travail réalisé dans le cadre de l'article \textit{Motion Planning Strategy for the Active Vision-Based Mapping of Ground-Level Structures}.

\section{Description du problème}
Une composante importante de tout système de SLAM est sa capacité de fermeture de boucle, c'est-à-dire d'être capable de reconnaître quand le robot retourne à un endroit déjà visité. Ceci peut être fait localement sur un sous-ensemble local des observations courantes ou globalement sur toutes les observations faites jusqu'au présent, par optimisation de graphe de poses \citep{Carlone2016} ou par compensation par faisceaux (\textit{Bundle Adjustment}) \citep{Mei2011} ou tel que proposé dans ORB-SLAM2 
 une combinaison des deux \citep{Mur-Artal2017}. La fermeture de boucle permet de minimiser les erreures de localisation 
\section{Exploration du périmètre}

\section{Exploration des cavités}

\section{Résultats}

\section{Conclusion}
